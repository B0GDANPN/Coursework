\documentclass[12pt]{article}
\usepackage{pdfpages}
\usepackage[a4paper, left=15mm, right=15mm, top=20mm, bottom=20mm]{geometry}

% Поддержка русского языка
\usepackage[utf8]{inputenc}
\usepackage[russian]{babel}
\renewcommand{\familydefault}{\sfdefault}

\usepackage{tikz} % Для рисования

\usepackage{amsmath} % Для псевдокода
\usepackage{amssymb}
\usepackage{mathtools}
\DeclarePairedDelimiter{\abs}{\lvert}{\rvert}
\usepackage{algorithm}
\usepackage{algpseudocode}
\usetikzlibrary{trees, positioning} % Подключаем библиотеку для работы с деревьями

\usepackage{array} % Для работы с таблицами
\usepackage{multirow}

\def\ojoin{\setbox0=\hbox{$\bowtie$} \rule[-.02ex]{.25em}{.4pt}\llap{\rule[\ht0]{.25em}{.4pt}}} % Для рисования внешних соединений
\def\leftouterjoin{\mathbin{\ojoin\mkern-5.8mu\bowtie}}
\def\rightouterjoin{\mathbin{\bowtie\mkern-5.8mu\ojoin}}
\def\fullouterjoin{\mathbin{\ojoin\mkern-5.8mu\bowtie\mkern-5.8mu\ojoin}}

\title{Методы ускорения перебора соединений таблиц в планировании SQL запросов} 
\author{НГУ, ММФ, 22126, Пьянзин Богдан}
\date{\today}

\begin{document}
\maketitle

\newpage


\section*{Структура курсовой работы}  % Звездочка убирает нумерацию заголовка
\addcontentsline{toc}{section}{Структура курсовой работы} % Добавляем в оглавление

\section{Введение}
\begin{itemize}
    \item Актуальность темы.
    \item Цель исследования.
    \item Задачи работы.
    \item Краткое описание рассматриваемых вопросов.
\end{itemize}

\section{Основная часть}

\subsection{Теоретические основы соединений таблиц в \textbf{SQL} запросах}
\begin{itemize}
    \item Определение и классификация соединений и планирование запроса.
    \item Задача выбора порядка \texttt{JOIN}-ов.
    \item Факторы, влияющие на производительность (размеры таблиц, статистики, индексы, типы данных и т.д.).
    \item Влияние порядка \texttt{JOIN} на производительность СУБД.
\end{itemize}

\subsection{Методы ускорения перебора соединений таблиц}
\begin{itemize}
    \item Классические подходы перебора соединений таблиц.
    \item Реализация в \textbf{PostgreSQL}.
    \item Современные подходы. \textcolor{red}{\%TODO}
\end{itemize}

\subsection{Сравнительный анализ методов ускорения \textcolor{red}{\%TODO}}
\begin{itemize}
    \item Преимущества и недостатки каждого метода \textcolor{red}{\%TODO}
\end{itemize}

\section{Заключение \textcolor{red}{\%TODO}} 
\begin{itemize}
    \item Итоги работы.
    \item Рекомендации по применению методов.
    \item Перспективы исследований.
\end{itemize}

\section{Список литературы}
\newpage
%%%%%%%%%%%%%%%%%%%%%%%%%%%%%%%%%%%%%%%%%%%%%%%%%%%%%%%%%%%%
\begin{center}
    \section*{Введение}
    \subsection*{Актуальность темы}
\end{center}
Современные СУБД работают с большим объёмом информации и транзакций, использую для ввода запросов язык SQL. 
В процессе трансляции запрос превращается сначала в логическое представление в виде дерева, 
затем с помощью оптимизатора СУБД в физический план исполнения. Производительность СУБД напрямую зависят от качества
сгенерированного физического плана. Для создания "хорошего" плана, нужно решить одну решить или приблизить решение
NP полной задачи перебора соединений таблиц,так как он требует сложных вычислений и значительных затрат ресурсов.
\newline
Сложность задачи обусловлена тем, что количество возможных соединений(бинарных) с n таблицами
равно количеству бинарных дерьев с n листьями - Числа Каталана, которые
имеют экспоненциальную скорость роста.
\newline
В классических подходах используется динамическое программирование(DP)
и эвристический поиск для решения этой проблемы. Однако с ростом количества таблиц,
потребление памяти( для хранения промежуточных результатов) и временя планирования становятся
слишком велики. Эвристические методы не гарантируют оптимальность решения, так как
находят локальное оптимальные планы, но не гарантируют глобальное оптимальное решение.
\newline
Современные подходы предлагают использовать машинное обучение на основе статистик,
которые могут динамически меняться в процессе выполнения запроса. И создавать на их
основе оптималььные планы.
\newline
Таким образом, выбор нееэффективного плана запроса может привести
к значительному замедлению работы системы, а процесс поиска хорошего решения является
нетривиальной задачей. В свою очередь оптимальный план позволяет эффективно исполнить
запрос, с приемлемым потреблением CPU, памяти, I/O, сетевых ресурсов(особенно актуально
для распределённых СУБД).

\end{document}
