\thispagestyle{empty}

\begin{center}
  МИНИСТЕРСТВО НАУКИ И ВЫСШЕГО ОБРАЗОВАНИЯ РОССИЙСКОЙ ФЕДЕРАЦИИ \\
  ФЕДЕРАЛЬНОЕ ГОСУДАРСТВЕННОЕ АВТОНОМНОЕ ОБРАЗОВАТЕЛЬНОЕ УЧРЕЖДЕНИЕ ВЫСШЕГО ОБРАЗОВАНИЯ \\
  «НОВОСИБИРСКИЙ НАЦИОНАЛЬНЫЙ ИССЛЕДОВАТЕЛЬСКИЙ ГОСУДАРСТВЕННЫЙ УНИВЕРСИТЕТ» \\
  (НОВОСИБИРСКИЙ ГОСУДАРСТВЕННЫЙ УНИВЕРСИТЕТ, НГУ)
\end{center}

\vspace{1.5cm}

\noindent % Отменяем отступ первой строки
Факультет \hfill Механико-математический \par

\vspace{1cm}

\noindent
Кафедра \hfill Программирования \par
\noindent
Направление подготовки \hfill Математика и компьютерные науки

\vspace{2cm}

\begin{center}
  \vspace{1em} % Отступ после линии
  {\Large \textbf{РЕФЕРАТ}} % Заголовок крупным и жирным шрифтом
  \vspace{1em} % Отступ перед линией
\end{center}

% \vspace{0.5cm}

\begin{center}
  \textbf{Пьянзин Богдан Олегович}
  \rule[0.5ex]{\linewidth}{0.4pt}
  \small (Фамилия, Имя, Отчество автора)
\end{center}

\vspace{1.5cm}

\noindent Тема: Методы адаптивного планирования запросов в реляционных СУБД

% \vfill % Растягиваем вертикальное пространство, чтобы следующий блок был ниже
\vspace{1.5cm}

% --- Блок утверждения и научного руководителя ---
\hspace*{\fill} % Убедимся, что minipage начнется от правого края
\begin{minipage}[t]{0.5\textwidth}
  \textbf{«Реферат принят»}
  
  \vspace{0.5cm} % Отступ
  
  \textbf{Научный руководитель}
  
  \vspace{0.5cm} % Отступ
  
  % Детали руководителя (можно выровнять по левому краю относительно центра или просто слева)
  % Здесь просто слева, но с небольшим отступом от центральной линии
  к.ф-м.н., с.н.с., \\
  старший преподаватель \\
  кафедры программирования ММФ
  
  \vspace{1cm} % Отступ
    % Подпись руководителя (по центру)
  \textbf{Пономарёв Д.К.} / \rule{4cm}{0.4pt} \\ % Фамилия И.О., слэш, линия для подписи
  \small (ФИО, Подпись) % Подсказка мелким шрифтом

  \vspace{1cm} % Отступ

  % Дата
  «\ldots»\makebox[7em]{\dotfill} 2025 г. % День, точки, год, г.
  % «\ldots» мая 2025 г. % День, точки, год, г.
\end{minipage}

\vfill % Растягиваем вертикальное пространство, чтобы последний блок был внизу

% --- Город и Год (по центру внизу) ---
\begin{center}
Новосибирск, 2025
\end{center}
